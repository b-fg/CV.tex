% LaTeX resume using res.cls
\documentclass[line]{res}
\usepackage{lmodern}
\usepackage[T1]{fontenc}
\usepackage{xcolor}
\usepackage{romannum}
\usepackage{fancyhdr}
\usepackage{etaremune}
\usepackage{refcount}
% \renewcommand{\labelenumi}{[\theenumi]} % etarenum from 1. to [1]
\usepackage[left=0.75in, right=1.25in,top=0.75in,bottom=0.75in]{geometry}
\usepackage{array}
\usepackage{hyperref}
\hypersetup{
	unicode=true, 
	pdftitle={CV Bernat Font},
	pdfauthor={Bernat Font},
	pdfnewwindow=true, 
	colorlinks=true, 	% (false,true)
	pdfborder={0 0 0},	
	linkcolor=blue,
	linktoc=all, 		% (none,all) 
	citecolor=blue,
	urlcolor=blue,
	breaklinks=false,
}
\usepackage{fancyhdr}
\usepackage{changepage}

\pagestyle{empty}  % No page numbers/headers/footers

\fancyhf{} % clear all header and footer fields
\renewcommand{\headrulewidth}{0pt} % no line in header area
\fancyfoot[L]{\hspace{-39pt} \footnotesize Last updated: \today}
%\fancyfoot[C]{\footnotesize \thepage}
\fancyfoot[R]{\footnotesize \thepage}

\pagestyle{fancy}

\newenvironment{p}
  {\begin{adjustwidth}{-10pt}{0pt}}
  {\end{adjustwidth}}

\newenvironment{p1}
  {\begin{adjustwidth}{-16pt}{0pt}
  \vspace{1pt}}
  {\end{adjustwidth}}

\newenvironment{p3}
  {\begin{adjustwidth}{-16pt}{0pt}
  \vspace{3pt}}
  {\end{adjustwidth}}
  
\begin{document}
\pagenumbering{arabic}

\name{\Large Dr. Bernat Font}

\begin{resume}
\section{}
\vspace{-15pt}
\hspace*{-36pt}{\href{mailto:bernatfontgarcia@gmail.com}{\texttt{bernatfontgarcia@gmail.com}}\hfill\href{https://b-fg.github.io}{\texttt{b-fg.github.io}}}\\
\hspace*{0pt}\hfill\,\,\href{https://github.com/b-fg}{\texttt{github.com/b-fg}}\\
\vspace{-25pt}
\noindent

\section{Research interests}
\begin{p}
Computational fluid dynamics, turbulence modelling, deep learning \& data-driven models, numerical methods, high-performance computing. 
\end{p}
 
\section{Experience}
\begin{p1}
\begin{tabular}{p{\dimexpr 0.85\linewidth-2\tabcolsep} >{\raggedleft\arraybackslash}p{\dimexpr 0.15\linewidth-\tabcolsep}}
\textbf{Postdoctoral Researcher}, Large-scale Computational Fluid Dynamics Group, Barcelona Supercomputing Center, Spain. & 2021-\\
\textit{Topic}: NextSim EU project -- Next generation of industrial aerodynamic simulation code. & \\
\\
\textbf{Postdoctoral Researcher}, Mathematical Institute, Oxford University, UK.& 2020-2021 \\
\textit{Topic}: Transport of porous particles in fluid flow. & \\
\\
\textbf{Visiting PhD Researcher}, Institute of High-Performance Computing, A*STAR, Singapore. & 2017-2020\\
\end{tabular}
\end{p1}

\section{Education}
\begin{p3}
\begin{tabular}{p{\dimexpr 0.87\linewidth-2\tabcolsep} >{\raggedleft\arraybackslash}p{\dimexpr 0.13\linewidth-\tabcolsep}}
\textbf{Ph.D.} Computational Fluid Dynamics, University of Southampton, UK. &  2015-2020\\
\textit{Thesis}: Modelling of Flow Past Long Cylindrical Structures. (\href{https://b-fg.github.io/assets/pdf/Font_2020_PhD_-_Modelling_of_Flow_Past_Long_Cylindrical_Structures.pdf}{\texttt{download}}) & \\
\textit{Supervisors}: Prof. G.D. Weymouth, Prof. O.R. Tutty, Dr. V.-T. Nguyen. & \\
\textit{Visiting Researcher}: Research attachment funded by the ARAP mobility scheme, Institute of High-Performance Computing, A*STAR, Singapore. &  \\
\\
\textbf{M.Sc.} Computational Fluid Dynamics, Cranfield University, UK. &  2014-2015\\ 
\textit{Thesis}: High-order Shock-capturing Schemes for Micro Shock Tubes. (\href{https://b-fg.github.io/assets/pdf/Font_2015_MSc_-_High-order_Shock-capturing_Schemes_for_Micro_Shock_Tubes.pdf}{\texttt{download}})& \\
\textit{Supervisor}: Dr. L. K\"{o}n\"{o}zsy. & \\
\textit{Double Degree with Ingenier\'{i}a Superior in Aeronautical Engineering}. & \\
\\
\textbf{Ingenier\'{i}a Superior} Aeronautical Engineering, Universitat Polit\`{e}cnica de Catalunya, Spain. &  2012-2015\\ 
\textit{Mentor}: Prof. C.-D. P\'{e}rez-Segarra. & \\
\textit{Equivalent to Master of Engineering}. & \\
\\
\textbf{Ingenier\'{i}a T\'{e}cnica} Aeronautical Engineering, Universitat Polit\`{e}cnica de Catalunya, Spain.&  2009-2012\\ 
\textit{Equivalent to Bachelor of Engineering}. & 
\end{tabular}
\end{p3}
 
\section{Publications}
\vspace{0.25cm}
\hspace{-1cm}\textbf{Peer-reviewed journal articles}\vspace{0.25cm}
\begin{p}
\begin{etaremune}[leftmargin=-2pt,parsep=5pt]
\item Varela, P., Suárez, P., Alcántara-Ávila, F., Miró, A., Rabault, J., Font, B., García-Cuevas, L.M., Lehmkuhl, O., Vinuesa, R (2022) Deep reinforcement learning for flow control exploits different physics for increasing Reynolds number regimes. \textit{Actuators} 11, 359. (\href{https://doi.org/10.3390/act11120359}{\texttt{doi}})
\item Font, B., Weymouth, G.D., Nguyen, V.-T. \& Tutty, O.R. (2021) Deep-learning of the spanwise-averaged Navier--Stokes equations. \textit{Journal of Computational Physics} 434, 110199. (\href{https://doi.org/10.1016/j.jcp.2021.110199}{\texttt{doi}} | \href{https://arxiv.org/abs/2008.07528}{\texttt{preprint}})
\item Font, B., Weymouth, G.D., Nguyen, V.-T. \& Tutty, O.R. (2019) Span effect on the turbulence nature of flow past a circular cylinder. \textit{Journal of Fluid Mechanics} 878, 306--323. (\href{https://doi.org/10.1017/jfm.2019.637}{\texttt{doi}} | \href{https://arxiv.org/abs/2008.08933}{\texttt{preprint}})
\end{etaremune}
\end{p}

\hspace{-1cm}\textbf{Peer-reviewed symposium proceedings}\vspace{0.25cm}
\begin{p}
\begin{etaremune}[leftmargin=-2pt,parsep=5pt]
\item Radhakrishnan, S., Gyamfi, L.A., Mir\'{o}, A., Font, B., Calafell, J., Lehmkuhl, O. \& Tutty, O.R. (2021) A data-driven wall-shear stress model for LES using gradient boosted decision trees \textit{ISC High Performance 2021} pp. 105--121. (\href{https://doi.org/10.1007/978-3-030-90539-2_7}{\texttt{doi}} | \href{https://upcommons.upc.edu/bitstream/handle/2117/358666/Manuscript_isc2021.pdf?sequence=3}{\texttt{preprint}})
\item Font, B., Weymouth, G.D., Nguyen, V.-T. \& Tutty, O.R. (2020) Turbulent wake prediction using deep convolutional neural networks \textit{33rd Symposium on Naval Hydrodynamics}, Osaka, Japan. (\href{https://eprints.soton.ac.uk/444591/}{\texttt{doi}})
\end{etaremune}
\end{p}

\hspace{-1cm}\textbf{Conference proceedings}\vspace{0.25cm}
\begin{p}
\begin{etaremune}[leftmargin=-2pt,parsep=5pt]
\item Font, B., Weymouth, G.D.  \&  Tutty, O.R. (2017) Analysis of two-dimensional and three-dimensional wakes of long circular cylinders. {\em OCEANS MTS/IEEE}, Aberdeen, UK. (\href{https://doi.org/10.1109/OCEANSE.2017.8084904}{\texttt{doi}})
\end{etaremune}
\end{p}

\hspace{-1cm}\textbf{Abstracts}\vspace{0.25cm}
\begin{p}
\begin{etaremune}[leftmargin=0pt,parsep=5pt]
\item Font, B., Mir\'{o}, A. \&  Lehmkuhl, O. (2023)  On the entropy-viscosity method for flux reconstruction. \textit{2nd Spanish Fluid Mechanics Conference}, Barcelona, Spain. [submitted]
\item Font, B., Weymouth, G.D.  \&  Tutty, O.R. (2019)  Deep learning the spanwise-averaged wake of a circular cylinder. \textit{72nd Meeting of the APS Division of Fluid Dynamics}, Seattle, US. (\href{https://meetings.aps.org/Meeting/DFD19/Session/L17.5}{\texttt{url}} | \href{https://github.com/b-fg/APS2019}{\texttt{presentation}})
\item Font, B., Castells, I., Weymouth, G.D., Nguyen, V.-T.  \&  Tutty, O.R. (2019)  Turbulence dynamics transition of flow past a circular cylinder and the prediction of vortex-induced forces. \textit{European Turbulence Conference 17}, Torino, Italy. (\href{https://etc17.fyper.com/program/show_slot/41}{\texttt{url}})
\item Font, B., Weymouth, G.D.  \&  Tutty, O.R. (2016)  A two-dimensional model for three-dimensional symmetric flows. \textit{UK Fluids Conference}, London, UK. (\href{https://www.imperial.ac.uk/media/imperial-college/faculty-of-engineering/aeronautics/UK-Fluids-Conference-2016-booklet.pdf}{\texttt{url}})
\end{etaremune}
\end{p}

\section{Invited Talks}\vspace{0.5cm}
\begin{p}
PPPL Computer Science Department's Machine Learning seminar, Princeton University, September 2021.\\
Engineering Mind Podcast: CFD + Machine Learning, Turbulence \& PhD Life, June 2021. (\href{https://youtu.be/d1O3dFgvAP4}{\texttt{url}})\\
Applied Mathematics in Aerospace Engineering seminar, Universidad Politecnica de Madrid, Spain, April 2021.\\
Applied Math Colloquium, University North Carolina, US, March 2021.\\
Ocean Engineering, University Rhode Island, US, February 2021.\\
Fluid Dynamics Group Institute of High Performance Computing (A*STAR), Singapore, November 2020.\\
Fluid Structure Interactions Group seminar series, University of Southampton, UK, July 2020.
\end{p}

\section{Student supervision}\vspace{0.5cm}
\begin{p}
Mentored and supervised students at different stages of their educational program such as Undergraduate students, MSc students, and most recently a PhD student.
As a mentor, the goal is to motivate students to pursue an interesting scientific topic while providing guidance throughout the process of learning and achieving.
The supervision involves regular meetings to assess their progress and answering technical questions when needed.\\
\end{p}
\begin{p1}
\begin{tabular}{p{\dimexpr 0.85\linewidth-2\tabcolsep} >{\raggedleft\arraybackslash}p{\dimexpr 0.15\linewidth-\tabcolsep}}
Supervisor of PhD students, \textit{TU Delft} & 2023- \\
- Physics-based machine learning of marine hydrodynamics &\\
\\
Supervisor of undergraduate projects, \textit{Universitat Polit\`{e}cnica de Catalunya} & 2021- \\
- Discovering new scaling laws in turbulent boundary layers via multi-expression programming. (\href{http://hdl.handle.net/2117/372288}{\texttt{url}}) &\\
- Discovering new expressions for the vortex trajectories and velocity profiles of synthetic jets. (\href{http://hdl.handle.net/2117/365135}{\texttt{url}}) &\\
\\
Supervisor of MSc projects, \textit{University of Southampton} & 2019-2020 \\ 
- Machine learning wall model for bluff bodies forces calculation. &\\
- Accurate flow interpolation using optimal transport theory. &\\
\end{tabular}
\end{p1}

\section{Teaching}\vspace{0.5cm}
\begin{p}
Served as demonstrator and marker of multiple modules during my PhD.
Tasks involved preparing and delivering the laboratory sessions which included a theory part and an experimental part.
Additionally, served as lecturer of the BSC summer school on AI and HPC delivering a lecture on AI for CFD. \\
\end{p}
\begin{p1}
\begin{tabular}{p{\dimexpr 0.85\linewidth-2\tabcolsep} >{\raggedleft\arraybackslash}p{\dimexpr 0.15\linewidth-\tabcolsep}}
Lecturer at the PUMPS+AI Summer School, \textit{Barcelona Supercomputing Center} & 2022\\
- Machine learning for computational fluid dynamics (\href{https://pumps.bsc.es/2022/}{\texttt{url}})\\
\\
Demonstrator, \textit{University of Southampton} & 2015-2017 \\ 
- Aerodynamics: Nozzle lab. &\\
- Propulsion: Ramjet, turbojet and rocket engine labs. &\\
- Aerothermodynamics: Marking of lab reports. &
\end{tabular}
\end{p1}

\section{Software skills}\vspace{0.5cm}
\begin{p}\setlength{\parskip}{3pt}
\textbf{Programming languages}: Fortran (including MPI), Julia, Python (including PyTorch, Keras, and Tensorflow), C, Java, Matlab. \\
\textbf{Tools}: Git, \LaTeX, Inkscape, Paraview.
\end{p}


\section{Open science statement}\vspace{0.5cm}
\begin{p}\setlength{\parskip}{3pt}
I advocate for open-science.
Most of my papers have an e-print version that can be downloaded for free either on arXiv or \href{https://b-fg.github.io/}{my website.}
The codes I write are also open-source, and you can find them in my \href{https://github.com/b-fg}{Github repository}.
\end{p}


\section{References}\vspace{0.2cm}
\begin{p1}\setlength{\parskip}{1em}
Gabriel D. Weymouth, Professor of Ship Hydromechanics.\newline
TU Delft, Netherlands.\newline
\href{mailto:g.d.weymouth@tudelft.nl}{\texttt{g.d.weymouth@tudelft.nl}}

Carles-David P\'{e}rez-Segarra, Professor at the Heat and Mass Transfer Technological Center.\newline
Universitat Polit\`{e}cnica de Catalunya, Spain.\newline
\href{mailto:segarra@cttc-upc.net}{\texttt{segarra@cttc-upc.net}}

Vinh-Tan Nguyen, Senior Scientist at the Institute of High Performance Computing.\newline
A*STAR, Singapore.\newline
\href{mailto:nguyenvt@ihpc.a-star.edu.sg}{\texttt{nguyenvt@ihpc.a-star.edu.sg}}

F. Xavier Trias, Associate Professor at the Heat and Mass Transfer Technological Center.\newline
Universitat Polit\`{e}cnica de Catalunya, Spain.\newline
\href{mailto:xavi@cttc.upc.edu}{\texttt{xavi@cttc.upc.edu}}

Owen R. Tutty, Professor at the Aerodynamics and Flight Mechanics Group.\newline
University of Southampton, UK.\newline
\href{mailto:o.r.tutty@soton.ac.uk}{\texttt{o.r.tutty@soton.ac.uk}}
\end{p1}
% \section{Professional Societies}
% \section{Awards \& Honors}
% \section{Relevant Skills}
\end{resume}
\end{document}







