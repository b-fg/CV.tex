% LaTeX resume using res.cls
\documentclass[line]{res}
\usepackage{lmodern}
\usepackage[T1]{fontenc}
\usepackage{xcolor}
\usepackage{romannum}
\usepackage{fancyhdr}
\usepackage{etaremune}
\usepackage{refcount}
\usepackage[left=0.75in, right=1.25in,top=0.75in,bottom=0.75in]{geometry}
\usepackage{array}
\usepackage{amssymb}
\usepackage{fancyhdr}
\usepackage{changepage}
\usepackage{longtable}
\usepackage{graphicx}
\usepackage{bibentry}
\usepackage{natbib}
\usepackage[resetlabels]{multibib}
% \usepackage[backend=biber]{biblatex}
% \addbibresource{main.bib}

\usepackage{hyperref}
\hypersetup{
	unicode=true,
	pdftitle={CV Bernat Font},
	pdfauthor={Bernat Font},
	pdfnewwindow=true,
	colorlinks=true, 	% (false,true)
	pdfborder={0 0 0},
	linkcolor=blue,
	linktoc=all, 		% (none,all)
	citecolor=blue,
	urlcolor=blue,
	breaklinks=false,
}
\pagestyle{empty}  % No page numbers/headers/footers

\fancyhf{} % clear all header and footer fields
\renewcommand{\headrulewidth}{0pt} % no line in header area
\fancyfoot[L]{\hspace{-39pt} \footnotesize Last updated: \today}
%\fancyfoot[C]{\footnotesize \thepage}
\fancyfoot[R]{\footnotesize \thepage}

\pagestyle{fancy}

\newenvironment{p1}
{\begin{adjustwidth}{-30pt}{0pt}
\vspace{8pt}}
{\end{adjustwidth}}

\newenvironment{p11}
{\begin{adjustwidth}{-25pt}{0pt}
\vspace{8pt}}
{\end{adjustwidth}}

\newenvironment{p2}
{\begin{adjustwidth}{-30pt}{0pt}
\vspace{10pt}}
{\end{adjustwidth}}

\newcommand\sbullet[1][.5]{\mathbin{\vcenter{\hbox{\scalebox{#1}{$\bullet$}}}}}
% \renewcommand{\labelitemi}{$\square$}
\renewcommand{\labelitemi}{$\sbullet[.75]$}
% \newcommand{\tabitem}{$\square$~~}
\newcommand{\tabitem}{$\sbullet[.75]$~~}
\renewcommand{\arraystretch}{1.25}

\begin{document}
\pagenumbering{arabic}
\nobibliography{main}
\bibliographystyle{apalike-refs}

\name{\Large Dr. Bernat Font}

\begin{resume}
\section{}
\vspace{-15pt}
\hspace*{-36pt}{\href{mailto:b.font@tudelft.nl}{\texttt{b.font@tudelft.nl}}\hfill\href{https://b-fg.github.io}{\texttt{b-fg.github.io}}}\\
\hspace*{0pt}\hfill\,\,\href{https://github.com/b-fg}{\texttt{github.com/b-fg}}\\
\vspace{-25pt}
\noindent

\section{Research interests}
\begin{p11}
computational fluid dynamics; machine learning; turbulence modelling; numerical methods; high-performance computing.
\end{p11}

\section{Experience}

\begin{p1}
\begin{tabular}{p{\dimexpr 0.89\linewidth-2\tabcolsep} >{\raggedleft\arraybackslash}p{\dimexpr 0.11\linewidth-\tabcolsep}}
	\textbf{Assistant Professor}, Ship Hydromechanics section, Technische Universiteit Delft, Netherlands. & 2024--\\
	\textit{Focus}: Data-Informed CFD, i.e. data exploitation to improve current CFD methods:
	\begin{itemize}
		\item Acceleration of scale-resolving simulations using generative methods
		\item Data-driven wall models for non-equilibrium turbulent boundary layers
		\item Active flow control using reinforcement learning
		\item Discovery of governing equations from raw data
	\end{itemize}
\end{tabular}

\vspace{-5pt}
\begin{tabular}{p{\dimexpr 0.89\linewidth-2\tabcolsep} >{\raggedleft\arraybackslash}p{\dimexpr 0.11\linewidth-\tabcolsep}}
	\textbf{Postdoctoral Researcher}, Large-scale Computational Fluid Dynamics group, Barcelona Supercomputing Center, Spain. & 2021--2023 \\
	\textit{Topics}:
	\begin{itemize}
		\item NextSim EU project -- Next generation of industrial aerodynamic simulation code
		\item Machine learning for CFD: wall models and active flow control
		\item Turbulence modelling in high-order scale-resolving simulations
	\end{itemize}
\end{tabular}

\vspace{-5pt}
\begin{tabular}{p{\dimexpr 0.89\linewidth-2\tabcolsep} >{\raggedleft\arraybackslash}p{\dimexpr 0.11\linewidth-\tabcolsep}}
	\textbf{Researcher}, Mathematical Institute, Oxford University, UK.& 2020--2021 \\
\textit{Topic}: Transport of porous particles in fluid flow. & \\
\end{tabular}

\vspace{5pt}
\begin{tabular}{p{\dimexpr 0.89\linewidth-2\tabcolsep} >{\raggedleft\arraybackslash}p{\dimexpr 0.11\linewidth-\tabcolsep}}
	\textbf{Visiting PhD Researcher}, Institute of High-Performance Computing, A*STAR, Singapore. & 2017--2020 \\
\end{tabular}
\end{p1}

\section{Education}

\begin{p1}
\begin{tabular}{p{\dimexpr 0.89\linewidth-2\tabcolsep} >{\raggedleft\arraybackslash}p{\dimexpr 0.11\linewidth-\tabcolsep}}
	\textbf{Ph.D.} Computational Fluid Dynamics, University of Southampton, UK. & 2015--2021 \\
	\textit{Thesis}: Modelling of Flow Past Long Cylindrical Structures. (\href{https://b-fg.github.io/assets/pdf/Font_2020_PhD_Modelling_of_Flow_Past_Long_Cylindrical_Structures.pdf}{\texttt{eprint}}) & \\
	\textit{Supervisors}: Prof. G.D. Weymouth, Prof. O.R. Tutty, Dr. V.-T. Nguyen. & \\
	\textit{Visiting Researcher}: Research attachment funded by the ARAP mobility scheme, Institute of High-Performance Computing, A*STAR, Singapore. & \\
\end{tabular}

\vspace{5pt}
\begin{tabular}{p{\dimexpr 0.89\linewidth-2\tabcolsep} >{\raggedleft\arraybackslash}p{\dimexpr 0.11\linewidth-\tabcolsep}}
	\textbf{M.Sc.} Computational Fluid Dynamics, Cranfield University, UK. &  2014--2015\\
\textit{Thesis}: High-order Shock-capturing Schemes for Micro Shock Tubes. (\href{https://b-fg.github.io/assets/pdf/Font_2015_MSc_High-order_Shock-capturing_Schemes_for_Micro_Shock_Tubes.pdf}{\texttt{eprint}})& \\
\textit{Supervisor}: Dr. L. K\"{o}n\"{o}zsy. & \\
\textit{Double Degree with Enginyeria Superior in Aeronautical Engineering}. & \\
\end{tabular} \\

\vspace{5pt}
\begin{tabular}{p{\dimexpr 0.89\linewidth-2\tabcolsep} >{\raggedleft\arraybackslash}p{\dimexpr 0.11\linewidth-\tabcolsep}}
	\textbf{Enginyeria Superior} Aeronautical Engineering, Universitat Polit\`{e}cnica de Catalunya, Spain. & 2012--2015 \\
\textit{Mentor}: Prof. C.-D. P\'{e}rez-Segarra. & \\
\textit{Equivalent to Master of Engineering}. & \\
\end{tabular} \\

\vspace{5pt}
\begin{tabular}{p{\dimexpr 0.89\linewidth-2\tabcolsep} >{\raggedleft\arraybackslash}p{\dimexpr 0.11\linewidth-\tabcolsep}}
	\textbf{Enginyeria Tècnica} Aeronautical Engineering, Universitat Polit\`{e}cnica de Catalunya, Spain.& 2009--2012 \\
\textit{Equivalent to Bachelor of Engineering}. &
\end{tabular}
\end{p1}

\break

\section{Publications}
\begin{p1}
\textbf{Peer-reviewed journal articles}
\begin{etaremune}
    \item \bibentry{Suarez2025a}
    \item \bibentry{Garcia2025}
    \item \bibentry{Cutz2025}
    \item \bibentry{Font2025}
    \item \bibentry{WeymouthFont2024}
    \item \bibentry{Suarez2024a}
    \item \bibentry{Radhakrishnan2024}
    \item \bibentry{Varela2022}
    \item \bibentry{Font2021}
    \item \bibentry{Font2019}
\end{etaremune}

\textbf{Peer-reviewed symposium proceedings}
\begin{etaremune}
    \item \bibentry{Font2024a}
    \item \bibentry{Radhakrishnan2021}
    \item \bibentry{Font2020a}
\end{etaremune}

\textbf{Conference proceedings}
\begin{etaremune}
    \item \bibentry{Cabral2024ECCOMAS}
    \item \bibentry{Montala2024ECCOMAS}
    \item \bibentry{Suarez2023ERCOFTAC}
	\item \bibentry{Weymouth2023ParCFD}
    \item \bibentry{Font2017OCEANS}
\end{etaremune}

\textbf{Conference abstracts}
\begin{etaremune}
    \item \bibentry{Font2024JuliaCon}
    \item \bibentry{Font2024ECCOMAS}
    \item \bibentry{Cabral2024ECCOMASa}
    \item \bibentry{Montala2024ECCOMASa}
    \item \bibentry{Cabral2023APS}
    \item \bibentry{Alcantara-Avila2023APS}
    \item \bibentry{Weymouth2023APS}
    \item \bibentry{Alcantara-Avila2023ETC}
    \item \bibentry{Suarez2023M2P}
    \item \bibentry{Font2023SFMC}
    \item \bibentry{Font2022HiFiLED}
    \item \bibentry{Font2022ECCOMAS}
    \item \bibentry{Font2019APS}
    \item \bibentry{Font2019ETC}
    \item \bibentry{Font2016UKFLUIDS}
\end{etaremune}

\section{Invited Talks}
\begin{etaremune}
    \item \bibentry{talk12}
    \item \bibentry{talk11}
    \item \bibentry{talk10}
    \item \bibentry{talk9}
    \item \bibentry{talk8}
    \item \bibentry{talk7}
    \item \bibentry{talk6}
    \item \bibentry{talk5}
    \item \bibentry{talk4}
    \item \bibentry{talk3}
    \item \bibentry{talk2}
    \item \bibentry{talk1}
\end{etaremune}
\end{p1}

\section{Grants}
\begin{p1}
\begin{etaremune}
    \item \bibentry{DEAREL}
\end{etaremune}
\end{p1}

\section{Student supervision}\vspace{0.5cm}
\begin{p11}
Mentored and supervised students at different stages of their educational program such as Undergraduate students, MSc students, and most recently a PhD student.
As a mentor, the goal is to motivate students to pursue an interesting scientific topic while providing guidance throughout the process of learning and achieving.
The supervision involves regular meetings to assess their progress, and answering technical questions when needed. \\
\end{p11}

\vspace{-15pt}
\begin{p11}
\textbf{PhD students}

\vspace{5pt}
\begin{tabular}{p{\dimexpr 0.89\linewidth-2\tabcolsep} >{\raggedleft\arraybackslash}p{\dimexpr 0.11\linewidth-\tabcolsep}}
	\tabitem Wall-modelled large-eddy simulation for wind-assisted ships, TU Delft & 2025--\\
	\tabitem P. Muñoz, Non-equilibrium wall modelling using machine learning, Barcelona Supercomputing Center & 2024--\\
	\tabitem M. Cabral, Physics-based machine learning of marine hydrodynamics, TU Delft & 2023--\\
\end{tabular}\\

\textbf{MSc projects}

\vspace{5pt}
\begin{tabular}{p{\dimexpr 0.89\linewidth-2\tabcolsep} >{\raggedleft\arraybackslash}p{\dimexpr 0.11\linewidth-\tabcolsep}}
	\tabitem Algebraic multi-grid solver optimization, TU Delft & 2025 \\
	\tabitem Tandem Flettner rotors modelling, TU Delft & 2025 \\
	\tabitem Prediction of hydrofoil ventilation onset using temporal forecasting, TU Delft & 2025 \\
	\tabitem Machine-learning wall model for bluff bodies forces calculation, University of Southampton & 2019 \\
	\tabitem Accurate flow interpolation using optimal transport theory, University of Southampton & 2018 \\
\end{tabular}\\

\textbf{Undergraduate projects}

\vspace{5pt}
\begin{tabular}{p{\dimexpr 0.89\linewidth-2\tabcolsep} >{\raggedleft\arraybackslash}p{\dimexpr 0.11\linewidth-\tabcolsep}}
	\tabitem Discovering new scaling laws in turbulent boundary layers via multi-expression programming, Universitat Polit\`{e}cnica de Catalunya (\href{http://hdl.handle.net/2117/372288}{\texttt{url}}) & 2021 \\
	\tabitem Discovering new expressions for the vortex trajectories and velocity profiles of synthetic jets, Universitat Polit\`{e}cnica de Catalunya (\href{http://hdl.handle.net/2117/365135}{\texttt{url}}) & 2021 \\
\end{tabular} \\
\end{p11}

\section{Teaching}
\begin{p11}
Lecturer at TU Delft. Also, served as demonstrator and marker of multiple modules during my PhD.
Demonstration-related tasks involved preparing and delivering the laboratory sessions which included a theory part and an experimental part.
Additionally, served as lecturer of the BSC summer school on AI and HPC delivering a lecture on AI for CFD. \\

\begin{tabular}{p{\dimexpr 0.89\linewidth-2\tabcolsep} >{\raggedleft\arraybackslash}p{\dimexpr 0.11\linewidth-\tabcolsep}}
	Lecturer at TU Delft & 2024-- \\
\end{tabular}
\begin{itemize}
	\item Instructor at Hydromechanics (MT2461): Hydrofoils
	\item Invited lecturer at Numerical Ship Hydrodynamics (MT44025): Turbulence modelling
\end{itemize}

\begin{tabular}{p{\dimexpr 0.89\linewidth-2\tabcolsep} >{\raggedleft\arraybackslash}p{\dimexpr 0.11\linewidth-\tabcolsep}}
	Lecturer at the PUMPS+AI Summer School, Barcelona Supercomputing Center & 2022 \\
\end{tabular}
\begin{itemize}
	\item Machine learning for computational fluid dynamics (\href{https://pumps.bsc.es/2022/}{\texttt{url}})
\end{itemize}

\begin{tabular}{p{\dimexpr 0.89\linewidth-2\tabcolsep} >{\raggedleft\arraybackslash}p{\dimexpr 0.11\linewidth-\tabcolsep}}
	Demonstrator, University of Southampton & 2015--2017 \\
\end{tabular}
\begin{itemize}
	\item Aerodynamics: Nozzle lab
	\item Propulsion: Ramjet, turbojet and rocket engine labs
	\item Aerothermodynamics: Marking of lab reports
\end{itemize}
\end{p11}

\section{Software}
\begin{p11}
\textbf{Programming languages}: Fortran (including MPI), Julia, Python (including PyTorch, Keras, and Tensorflow), C, Java, Matlab.

\vspace{4pt}
\textbf{Tools}: Git, \LaTeX, Inkscape, Paraview.

\vspace{4pt}
\textbf{Selection of popular repositories}:
\begin{itemize}
	\item \bibentry{Font2024SmartSOD2D}
	\item \bibentry{Weymouth2023WaterLily}
	\item \bibentry{Font2020SANSpy}
	\item \bibentry{Font2019f2py}
	\item \bibentry{Font2018nuatsbot}
	\item \bibentry{Font2016PostProc}
\end{itemize}
\end{p11}

\section{Open science statement}
\begin{p11}\setlength{\parskip}{3pt}
I advocate for open science.
Most of my papers have an e-print version that can be downloaded for free either on \href{https://arxiv.org/search/physics?searchtype=author&query=Font%2C+B}{arXiv} or \href{https://b-fg.github.io/}{my website.}
The codes I write are also open-source, and you can find them in my \href{https://github.com/b-fg}{Github repository}.
\end{p11}

\end{resume}
\end{document}







